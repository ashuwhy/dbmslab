\documentclass[a4paper,12pt]{article}
\usepackage[utf8]{inputenc}
\usepackage[T1]{fontenc}
\usepackage{newtxtext,newtxmath} % Times New Roman style for text and math
\usepackage[margin=1in]{geometry} % Standard 1-inch margins
\usepackage{booktabs} % Professional academic tables
\usepackage{tabularx} % Auto-width tables
\usepackage{enumitem} % Better list control
\usepackage{titlesec} % Section heading formatting
\usepackage{fancyhdr} % Headers and footers
\usepackage{hyperref} % Hyperlinks
\usepackage{graphicx} % For including images
\usepackage{placeins} % For \FloatBarrier

% Header and Footer setup
\pagestyle{fancy}
\fancyhf{}
\lhead{\small \textit{DBMS Lab -- Assignment IV}}
\rhead{\small \textit{Student Academic Management System}}
\cfoot{\thepage}
\renewcommand{\headrulewidth}{0.4pt}

% Section heading adjustments (Compact and bold)
\titleformat{\section}{\large\bfseries}{\thesection}{1em}{}
\titleformat{\subsection}{\normalsize\bfseries}{\thesubsection}{1em}{}
\titleformat{\subsubsection}{\normalsize\itshape}{\thesubsubsection}{1em}{}

% Table setup
\newcolumntype{L}{>{\raggedright\arraybackslash}X} % Left-aligned scalable column
\newcolumntype{P}[1]{>{\raggedright\arraybackslash}p{#1}} % Fixed width, left aligned

\begin{document}

% Title Block
\begin{center}
    \vspace*{0.5cm}
    {\LARGE \textbf{Index Corruption -- Assignment IV}} \\[0.5em]
    {\large DBMS Lab -- Assignment IV} \\[1.5em]
\end{center}

\noindent This document describes the group, schema design, and technology stack for the Student Academic Management System (DBMS Lab Assignment IV).

\section{Group Name and Members}

\noindent \textbf{Group Name:} Index Corruption

\noindent \textbf{Members:}
\begin{itemize}[noitemsep, topsep=0pt]
    \item Ashutosh Sharma (23CS10005)
    \item Sujal Anil Kaware (23CS30056)
    \item Parag Mahadeo Chimankar (23CS10049)
    \item Kshetrimayum Abo (23CS30029)
    \item Kartik Pandey (23CS30026)
\end{itemize}

\section{Schema Design}

The system uses a PostgreSQL database with the following tables. The schema supports universities, programs, courses, students, instructors, enrollments, teaching assignments, content items, and role-based access control.

\subsection{RBAC and Users}

\begin{table}[h]
    \centering
    \renewcommand{\arraystretch}{1.2}
    \begin{tabularx}{\textwidth}{@{} l L @{}}
        \toprule
        \textbf{Table} & \textbf{Description} \\
        \midrule
        \texttt{app\_user} & Roles-based access control: \texttt{id} (PK), \texttt{email} (unique), \texttt{password\_hash}, \texttt{role} (admin / instructor / student / analyst), \texttt{created\_at}. Links to \texttt{student.email} or \texttt{instructor.email}. \\
        \bottomrule
    \end{tabularx}
\end{table}

\subsection{Core Domain Tables}

\begin{table}[h]
    \centering
    \renewcommand{\arraystretch}{1.2}
    \begin{tabularx}{\textwidth}{@{} l L @{}}
        \toprule
        \textbf{Table} & \textbf{Key Columns} \\
        \midrule
        \texttt{university} & \texttt{university\_id} (PK), \texttt{name} (unique), \texttt{country} \\
        \texttt{program} & \texttt{program\_id} (PK), \texttt{program\_name}, \texttt{program\_type}, \texttt{duration\_weeks\_or\_months} \\
        \texttt{course} & \texttt{course\_id} (PK), \texttt{course\_name} (unique), \texttt{duration\_weeks}, \texttt{university\_id} (FK), \texttt{program\_id} (FK), \texttt{textbook\_id} (FK) \\
        \texttt{textbook} & \texttt{textbook\_id} (PK), \texttt{title}, \texttt{isbn} (unique), \texttt{url} \\
        \texttt{topic} & \texttt{topic\_id} (PK), \texttt{topic\_name} (unique) \\
        \texttt{course\_topic} & \texttt{course\_id} (FK, PK), \texttt{topic\_id} (FK, PK) \\
        \bottomrule
    \end{tabularx}
\end{table}

\subsection{People and Assignments}

\begin{table}[h]
    \centering
    \renewcommand{\arraystretch}{1.2}
    \begin{tabularx}{\textwidth}{@{} l L @{}}
        \toprule
        \textbf{Table} & \textbf{Key Columns} \\
        \midrule
        \texttt{student} & \texttt{student\_id} (PK), \texttt{email} (unique), \texttt{full\_name}, \texttt{age} ($\ge$ 13), \texttt{country}, \texttt{category}, \texttt{skill\_level} \\
        \texttt{instructor} & \texttt{instructor\_id} (PK), \texttt{full\_name}, \texttt{email} (unique) \\
        \texttt{enrollment} & \texttt{student\_id} (FK, PK), \texttt{course\_id} (FK, PK), \texttt{enroll\_date}, \texttt{evaluation\_score} \\
        \texttt{teaching\_assignment} & \texttt{instructor\_id} (FK, PK), \texttt{course\_id} (FK, PK), \texttt{role} \\
        \texttt{content\_item} & \texttt{content\_id} (PK), \texttt{course\_id} (FK), \texttt{content\_type}, \texttt{title}, \texttt{url} \\
        \bottomrule
    \end{tabularx}
\end{table}

\subsection{Relationships (Summary)}
\begin{itemize}[noitemsep]
    \item \textbf{Course} belongs to one University and one Program; has one Textbook; many ContentItems; many Enrollments; many TeachingAssignments; many Topics via \texttt{course\_topic}.
    \item \textbf{Student} has many Enrollments; email links to \texttt{app\_user} for login.
    \item \textbf{Instructor} has many TeachingAssignments; email links to \texttt{app\_user} for login.
    \item \textbf{Enrollment} is a many-to-many between Student and Course with \texttt{enroll\_date} and \texttt{evaluation\_score}.
    \item \textbf{Teaching assignment} is a many-to-many between Instructor and Course with \texttt{role}.
\end{itemize}

\section{Entity-Relationship Diagram}

The complete ER diagram for the Student Academic Management System is shown below.

\begin{figure}[!htbp]
    \centering
    \includegraphics[width=\textwidth,keepaspectratio]{er.jpeg}
    \caption{Entity-Relationship Diagram for Student Academic Management System}
    \label{fig:er-diagram}
\end{figure}

% IMPORTANT: prevents the figure from floating past this point
\FloatBarrier

\section{Languages and Technology Stack}

\subsection{Overview}
\begin{itemize}[noitemsep]
    \item \textbf{Frontend:} Next.js (App Router), React, TypeScript
    \item \textbf{Backend:} FastAPI (Python 3.11+)
    \item \textbf{Database:} PostgreSQL
    \item \textbf{Build/Tooling:} Make, uv (or pip), npm
\end{itemize}

\subsection{Backend (API)}
\begin{itemize}[noitemsep]
    \item \textbf{Language:} Python 3.11+
    \item \textbf{Framework:} FastAPI
    \item \textbf{ORM / DB:} SQLAlchemy 2.x (async), asyncpg (PostgreSQL driver)
    \item \textbf{Validation \& Settings:} Pydantic, pydantic-settings
    \item \textbf{Auth:} JWT (python-jose), password hashing (passlib/bcrypt)
    \item \textbf{Server:} Uvicorn
\end{itemize}

\subsection{Frontend (Web)}
\begin{itemize}[noitemsep]
    \item \textbf{Language:} TypeScript
    \item \textbf{Framework:} Next.js 16 (App Router)
    \item \textbf{UI:} React 19, Tailwind CSS 4, Radix UI (dialogs, labels, scroll-area, etc.), Lucide React (icons)
    \item \textbf{Linting:} ESLint, eslint-config-next
\end{itemize}

\subsection{Project Structure}
\begin{itemize}[noitemsep]
    \item \texttt{/apps/api} -- FastAPI backend (routers: auth, student, instructor, admin, analyst)
    \item \texttt{/apps/web} -- Next.js frontend (admin, analyst, instructor, student, login)
    \item \texttt{/docs} -- Documentation (API.md, SCHEMA.md) and this submission document
\end{itemize}

\section{Functionalities and Workflows}

The system implements a complete academic management platform with four user roles, each with distinct capabilities and dedicated frontend pages.

\subsection{Roles}

The platform defines four roles, each enforced via JWT-based Role-Based Access Control (RBAC):

\begin{enumerate}[noitemsep]
    \item \textbf{Student} -- Browse courses, apply for enrollment, view course materials, and check grades.
    \item \textbf{Instructor} -- Manage assigned courses, create content, grade students, propose new courses and topics, and view course analytics.
    \item \textbf{Analyst} -- Access platform-wide data, generate performance reports, and analyze trends.
    \item \textbf{Admin} -- Approve registrations, manage users, approve course/topic proposals, and oversee all data.
\end{enumerate}

\subsection{Registration and Approval Workflow}

Each role follows a distinct onboarding process:

\subsubsection{Student Registration}
Students register by providing name, email, age ($\ge$ 13), country, skill level, and password. A corresponding \texttt{student} record is created immediately. No admin approval is required; students can log in and use the platform right away.

\subsubsection{Instructor Registration}
Instructors register with name, email, years of teaching experience, and password. An \texttt{app\_user} record is created with \texttt{approved\_at = NULL} and a linked \texttt{instructor} record is inserted. The instructor \textbf{cannot log in} until an admin sets \texttt{approved\_at} via the admin dashboard.

\subsubsection{Analyst Registration}
Analysts register with name, email, and password. Similar to instructors, their \texttt{app\_user} record has \texttt{approved\_at = NULL} and a linked \texttt{executive} record (type = analyst) is created. Admin approval is required before login.

\subsubsection{Admin}
Admins are bootstrapped or created by existing admins. They can also create users of any role directly from the admin panel.

\subsection{Student Module}

\begin{itemize}[noitemsep]
    \item \textbf{Browse Courses:} Students can search and filter available courses by name, topic, program type, university, and maximum duration.
    \item \textbf{Apply for Enrollment:} Students submit an enrollment application for a course. The enrollment is created with \texttt{status = pending} and a capacity slot is reserved.
    \item \textbf{My Applications:} Students can track the status of their pending or rejected applications.
    \item \textbf{My Enrollments:} Once approved by the instructor, students can view their enrolled courses, grades, and enrollment dates.
    \item \textbf{Course Detail Page:} For each enrolled course, students can view course materials (content items), textbook information, assigned instructors, topics, and their personal evaluation score.
    \item \textbf{Student Statistics:} Students can view aggregate stats such as number of courses enrolled, average score, highest score, and courses completed.
\end{itemize}

\subsection{Instructor Module}

\begin{itemize}[noitemsep]
    \item \textbf{Dashboard:} Displays all courses assigned to the instructor with student counts.
    \item \textbf{Student Management:} For each course, instructors can view enrolled students and pending applications. They can approve or reject student enrollment applications.
    \item \textbf{Grading:} Instructors can set or update evaluation scores (0--100) for enrolled students. All grade changes are logged in the \texttt{audit\_log} table for accountability.
    \item \textbf{Content Management:} Instructors can add content items (books, videos, notes with URLs) to their courses, and delete existing content.
    \item \textbf{Topic Management:} Instructors can link approved topics to their courses, or remove topic associations.
    \item \textbf{Course Proposals:} Instructors can propose new courses by specifying the name, duration, university, program, and textbook. The proposal enters a \texttt{pending} state and must be approved by an admin.
    \item \textbf{Topic Proposals:} Similarly, instructors can propose new topics for admin approval.
    \item \textbf{Course Analytics:} For each course, instructors can view score distribution (grade buckets A/B/C/D/F), pass rate, at-risk student count, average score, and total students.
    \item \textbf{Instructor Statistics:} Aggregate view of total courses taught, total students, overall average score, and highest/lowest course averages.
    \item \textbf{Textbook Management:} Instructors can create new textbook records for use in course proposals.
\end{itemize}

\subsection{Admin Module}

\begin{itemize}[noitemsep]
    \item \textbf{Dashboard:} Displays system-wide statistics (total users, courses, students, instructors, enrollments).
    \item \textbf{User Management:} List all users; create new users with any role (automatically creates linked \texttt{student}, \texttt{instructor}, or \texttt{executive} records as appropriate). Delete users with cascading removal of dependent records.
    \item \textbf{Registration Approvals:} View and approve/reject pending instructor and analyst registrations by setting the \texttt{approved\_at} timestamp.
    \item \textbf{Course Proposal Approvals:} Review pending course proposals from instructors. On approval, the system automatically creates the course, assigns the proposing instructor, and links specified topics. Proposals can also be rejected.
    \item \textbf{Topic Proposal Approvals:} Review and approve/reject topic proposals. On approval, a new \texttt{topic} record is created if it does not already exist.
    \item \textbf{Course Management:} Create courses directly, update course details (name, duration, university, program, topics), and assign or remove instructors from courses.
    \item \textbf{Student/Instructor Editing:} Update details for students (name, email, country, age, skill level) and instructors (name, email) via modal dialogs.
    \item \textbf{Enrollment Management:} Delete individual enrollments; the database trigger automatically updates the course's \texttt{current\_enrollment} counter.
    \item \textbf{University Management:} Create new university records for use in course creation.
\end{itemize}

\subsection{Analyst Module}

\begin{itemize}[noitemsep]
    \item \textbf{Overall Statistics:} Total courses, students, enrollments, and platform-wide average evaluation score.
    \item \textbf{Most Popular Course:} Identifies the course with the highest enrollment count, with optional university-based filtering.
    \item \textbf{Enrollments per Course:} Breakdown of enrollment counts for each course.
    \item \textbf{Average Score by Course:} Average evaluation scores ranked across all courses.
    \item \textbf{Top Indian Student by AI Average:} Finds the highest-performing Indian student in AI-related courses (uses \texttt{ILIKE} pattern matching on country and topic name).
    \item \textbf{Courses by University:} Course count distribution across universities.
    \item \textbf{Students by Country:} Geographic distribution of the student body.
    \item \textbf{Skill Level Distribution:} Breakdown of students by skill level (beginner, intermediate, advanced).
    \item \textbf{Top Courses:} Highest-enrollment courses with average scores.
\end{itemize}

\subsubsection{Advanced Reports (Reports Module)}
\begin{itemize}[noitemsep]
    \item \textbf{Module Analytics:} Cohort analysis showing average scores and student counts per program type.
    \item \textbf{Instructor Performance Index (IPI):} Computes the ratio of an instructor's average score to the global topic average, identifying high- and low-performing instructors per topic.
    \item \textbf{At-Risk Students:} Identifies students whose average evaluation score falls below a configurable threshold (default: 40).
    \item \textbf{Topic Trends:} Ranks topics by total enrollment volume to identify trending subject areas.
\end{itemize}

\subsection{Advanced DBMS Features}

The following advanced PostgreSQL features are utilized in the system:

\begin{itemize}[noitemsep]
    \item \textbf{Database Triggers:} A PL/pgSQL trigger (\texttt{trg\_auto\_enrollment\_count}) automatically increments or decrements \texttt{course.current\_enrollment} on enrollment insert or delete, ensuring data consistency without application-level logic.
    \item \textbf{Audit Logging:} Grade changes are recorded in the \texttt{audit\_log} table (old score, new score, changed by, timestamp), providing a complete audit trail. The audit log is viewable per course by instructors.
    \item \textbf{Window Functions:} Student rankings within a course are computed using PostgreSQL window functions (\texttt{RANK()}, \texttt{DENSE\_RANK()}, \texttt{PERCENT\_RANK()}, \texttt{ROW\_NUMBER()}) with \texttt{OVER (ORDER BY evaluation\_score DESC)}.
    \item \textbf{Pessimistic Locking:} A safe enrollment endpoint uses \texttt{SELECT ... FOR UPDATE} to lock the course row during enrollment, preventing race conditions when multiple students attempt to claim the last available seat simultaneously.
    \item \textbf{Check Constraints:} Enforced at the database level for \texttt{student.age} ($\ge 13$, $\le 100$), \texttt{enrollment.evaluation\_score} (0--100), and \texttt{course.duration\_weeks} ($> 0$).
    \item \textbf{Indexes:} Strategic indexes on frequently queried columns (\texttt{student.country}, \texttt{enrollment.evaluation\_score}, \texttt{course.duration\_weeks}, \texttt{course.course\_name}) to optimize query performance.
\end{itemize}

\end{document}